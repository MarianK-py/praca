% !TeX spellcheck = sk_SK-Slovak
\chapter*{Záver}  % chapter* je necislovana kapitola
\addcontentsline{toc}{chapter}{Záver} % rucne pridanie do obsahu
\markboth{Záver}{Záver} % vyriesenie hlaviciek

%Na záver už len odporúčania k samotnej kapitole Záver v bakalárskej práci podľa smernice \cite{smernica}:  \uv{V závere je potrebné v stručnosti zhrnúť dosiahnuté výsledky vo vzťahu k stanoveným cieľom. Rozsah záveru je minimálne dve strany. Záver ako kapitola sa nečísluje.}

%Všimnite si správne písanie slovenských úvodzoviek okolo predchádzajúceho citátu, ktoré sme dosiahli príkazom \verb'\uv'.

%V informatických prácach niekedy býva záver kratší ako dve strany, ale stále by to mal byť rozumne dlhý text, v rozsahu aspoň jednej strany. Okrem dosiahnutých cieľov sa zvyknú rozoberať aj otvorené problémy a námety na ďalšiu prácu v oblasti.

%Abstrakt, úvod a záver práce obsahujú podobné informácie. Abstrakt je kratší text, ktorý má pomôcť čitateľovi sa rozhodnúť, či vôbec prácu chce čítať. Úvod má umožniť zorientovať sa v práci skôr než ju začne čítať a záver sumarizuje najdôležitejšie veci po tom, ako prácu prečítal, môže sa teda viac zamerať na detaily a využívať pojmy zavedené v práci.

V tejto práci sme sa venovali problematike získavania medicínskych dát z prepúšťacej správy a krvných výsledkov. Našou snahou bolo vytvorenie systému ktorý by takéto získavanie dát zautomatizoval keďže nemocničný informačný systém Univerzitnej nemocnice v Bratislave (s ktorou sme na tomto projekte spolupracovali) neumožňoval jednoduché a rýchle získania štrukturovaných dát. 

Prvá kapitola je venovaná definovaniu metód využívaných na získavanie informácie z textu. Táto kapitola zároveň hovorí o troch už existujúcich systémov určených na problém získavania medicínskych dát a o rozdieloch medzi týmito systémami a naším systémom kvôli ktorým nebolo možné žiaden z týchto systémov priamo využiť pre naše dáta.

Druhá kapitola sa zaoberá vstupnými dátami, čiže textom z ktoré sa informácie snažíme získať. Konkrétna sa venuje výzoru týchto dát, ich obsahu čiže o tomu aké informácie tento text obsahuje a ktoré z nich sa mi snažíme získať. Taktiež sa táto kapitola venuje ak možným rozdeleniam tohto textu na menšie časti.

Tretia kapitola obsahuje jadro celej práce. Hovorí o postupoch využitých pre jednotlivé získavané informácie respektíve pre skupiny informácii. Zároveň hovorí o problémoch pri hľadaní týchto informácii ktoré sa počas vytvárania a nastavovania softvéru vyskytli a aké riešenia boli použité na ich odstránenie. V závere tejto kapitole sa nachádza informácia o výzore výstupných súborov ktorými sú tabuľka obsahujúca získané a logovací súbor obsahujúci informácie o chýbajúcich a problematických dátach. 

Obsahom štvrtej kapitoly je určenie chybovosti nášho systému inými slovami určenie ako často náš systém robí chyby. Túto chybovosť určujeme pre celý systém ako aj pre jednotlivé skupiny získavaných dát. Okrem chybovosti ktorú má náš systém sám o sebe určuje aj chybovosť po ľudskej kontrole a oprave výsledkov pomocou logovacieho súboru do ktorého si systém zaznamenáva nenájdené údaje a údaje pri ktorých získavaní nastal neočakávaný problém.

Piata kapitola je zameraná na vylepšenie systému tak aby bolo pre používateľa jednoduchšie upravovať množinu získavaných údajov. Snahou bolo aby používateľ potreboval na používanie a modifikáciu nášho systému čo najmenej vedomostí z oblasti informatiky konkrétne programovacieho jazyka Python a regulárnych výrazov. Tento cieľ sa doposiaľ podarilo spĺniť iba čiastočne pomocou oddelenia konfigurácie (množiny získavanej informácie) do samostatného súboru ktorý je písaný v človek ľahko čitateľnom formáte. Avšak stále je potrená znalosť aspoň základov o regulárnych výrazoch.  

