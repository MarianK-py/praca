% !TeX spellcheck = sk_SK-Slovak
\chapter*{Záver}  % chapter* je necislovana kapitola
\addcontentsline{toc}{chapter}{Záver} % rucne pridanie do obsahu
\markboth{Záver}{Záver} % vyriesenie hlaviciek

%Na záver už len odporúčania k samotnej kapitole Záver v bakalárskej práci podľa smernice \cite{smernica}:  \uv{V závere je potrebné v stručnosti zhrnúť dosiahnuté výsledky vo vzťahu k stanoveným cieľom. Rozsah záveru je minimálne dve strany. Záver ako kapitola sa nečísluje.}

%Všimnite si správne písanie slovenských úvodzoviek okolo predchádzajúceho citátu, ktoré sme dosiahli príkazom \verb'\uv'.

%V informatických prácach niekedy býva záver kratší ako dve strany, ale stále by to mal byť rozumne dlhý text, v rozsahu aspoň jednej strany. Okrem dosiahnutých cieľov sa zvyknú rozoberať aj otvorené problémy a námety na ďalšiu prácu v oblasti.

%Abstrakt, úvod a záver práce obsahujú podobné informácie. Abstrakt je kratší text, ktorý má pomôcť čitateľovi sa rozhodnúť, či vôbec prácu chce čítať. Úvod má umožniť zorientovať sa v práci skôr než ju začne čítať a záver sumarizuje najdôležitejšie veci po tom, ako prácu prečítal, môže sa teda viac zamerať na detaily a využívať pojmy zavedené v práci.

V tejto práci sme sa venovali problematike získavania medicínskych dát z prepúšťacej správy a krvných výsledkov. Našou snahou bolo vytvorenie systému ktorý by takéto získavanie dát zautomatizoval keďže nemocničný informačný systém Univerzitnej nemocnice v Bratislave (s ktorou sme na tomto projekte spolupracovali) neumožňoval jednoduché a rýchle získania štrukturovaných dát. Práca sa skladá z piatich kapitol. 

V prvej kapitola sme sa venovali definovaniu metód využívaných na získavanie informácie z textu. Táto kapitola zároveň hovorí o troch už existujúcich systémov určených na problém získavania medicínskych dát využívajúcich definované metódy a o rozdieloch medzi týmito systémami a naším systémom kvôli ktorým nebolo možné žiaden z týchto systémov priamo využiť pre naše zadanie.

V druhej kapitole sme sa zaoberali vstupnými dátami, čiže textom z ktorého sa informácie snažíme získať. Konkrétne sme sa venovali výzoru tohto textu a informáciám v ňom. V prípade výzoru sme skúmali štruktúru dát, konkrétne sme sa snažili rozdeliť dáta na viac častí pre rýchlejšie a presnejšie získavanie informácie. Z pohľadu informačného obsahu nás zaujímalo aké informácie dáta obsahujú, ktoré z nich potrebujeme získať a kde sa približne tieto informácie v dátach nachádzajú.   

Tretia kapitola obsahuje jadro celej práce. Hovorí o tom aký je všeobecný postup systému pri získavaní informácii a aj o postupoch využitých pri získavaní jednotlivých informácii respektíve skupín informácii. Konkrétnejšie ide o určenie kľúčových slov a všetkých bežných spôsobov zápisu danej informácie. Zároveň táto kapitola hovorí o problémoch ktoré sa vyskytli pri hľadaní týchto informácii počas vytvárania a nastavovania systému ako napríklad intervaly namiesto jednej číselnej hodnoty v prípade saturácie krvi kyslíkom alebo protichodné výsledky testov na protilátky proti SARS-CoV-2. Pre každý takýto problém sa tu nachádza aj riešenie ktoré bolo použité na jeho odstránenie. V závere tejto kapitole sa nachádza informácia o výzore výstupných súborov ktorými sú tabuľka obsahujúca získané dáta a logovací súbor obsahujúci informácie o chýbajúcich a problematických dátach. 

Vo štvrtej kapitole sme sa snažili zistiť rýchlosť a  chybovosť nášho systému inými slovami určenie toho či je náš systém rýchlejší ako človek a ako často náš systém robí chyby. Túto chybovosť určujeme pre celý systém ako aj pre jednotlivé skupiny získavaných dát. Okrem chybovosti ktorú má náš systém sám o sebe určuje aj chybovosť po ľudskej kontrole a oprave výsledkov pomocou logovacieho súboru do ktorého si systém zaznamenáva nenájdené údaje a údaje pri ktorých získavaní nastal problém. Zistili sme, že z pohľadu rýchlosti je náš systém výrazne rýchlejší ako ručné získavanie a to aj v prípade, že po získaní dát ich ručne opravujeme pomocou logovacieho súboru. Z pohľadu chybovosti sa ukázalo, že chybovosť systému samotného je porovnateľná s chybovosťou ručného získavanie, avšak po oprave použitím logovacieho súboru táto chybovosť klesá približne na polovicu. 

Piata kapitola je zameraná na vylepšenie systému tak aby bolo pre používateľa jednoduchšie upravovať množinu získavaných údajov. Snahou bolo aby používateľ potreboval na používanie a modifikáciu nášho systému čo najmenej vedomostí z oblasti informatiky konkrétne programovacieho jazyka Python a regulárnych výrazov. Tento cieľ sa doposiaľ podarilo spĺniť iba čiastočne pomocou oddelenia konfigurácie (množiny získavanej informácie) do samostatného súboru ktorý je písaný v človek ľahko čitateľnom formáte. Avšak stále je potrená znalosť aspoň základov o regulárnych výrazoch.

Finálnym výsledkom tejto práce je systém ktorý je schopný spoľahlivo získavať informácie o pacientoch s ochorením COVID-19 avšak je ho možné prekonfigurovať aj na získavanie informácii špecifických pre iné ochorenia ako sú špecifické krvné testy a predpisované lieky. Tento systém bol už aj použitý v praxi, dáta ním získané boli použité na analýzu v odbornom článku \cite{sabaka}. 