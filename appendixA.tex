% !TeX spellcheck = sk_SK-Slovak
\chapter*{Príloha A: obsah elektronickej prílohy}
\addcontentsline{toc}{chapter}{Príloha A}

%V elektronickej prílohe priloženej k práci sa nachádza zdrojový kód
%programu a súbory s výsledkami experimentov. Zdrojový kód je
%zverejnený aj na stránke \url{http://mojadresa.com/}.

%Ak uznáte za vhodné, môžete tu aj podrobnejšie rozpísať obsah tejto
%prílohy, prípadne poskytnúť návod na inštaláciu programu. Alternatívou
%je tieto informácie zahrnúť do samotnej prílohy, alebo ich uviesť na
%obidvoch miestach.

V elektronickej prílohe priloženej k práci sa nachádza zdrojový kód samotného systému v dvoch verziách a to ''Pacienti\_data\_finder.py'' čiže súbor vo formáte PY čo je klasický Python script a ''Pacienti\_data\_finder.ipynb'' čiže súbor vo formáte IPYNB čo je Interactive Python Notebook. Obsah týchto súborov je totožný, hlavným rozdielom je, že súbor vo formáte PY je vhodnejší pre používateľa vďaka jednoduchému spúšťaniu zatiaľ čo súbor vo formáte IPYNB je vhodnejší pre vývojára na testovanie drobných zmien v konkrétnych častiach kódu.

Ďalším súborom v prílohe je súbor ''informations.yml'' čo je konfiguračný súbor obsahujúci informáciu aké dáta získavame a aké špecifické regulárne výrazy na to používame. Tento súbor je vo formáte YAML.  

Obsah elektronickej prílohy je zverejnený aj na stránke \url{https://github.com/MarianK-py/elektronicka_priloha}.