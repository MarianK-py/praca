% !TeX spellcheck = sk_SK-Slovak
\chapter*{Úvod} % chapter* je necislovana kapitola
\addcontentsline{toc}{chapter}{Úvod} % rucne pridanie do obsahu
\markboth{Úvod}{Úvod} % vyriesenie hlaviciek

V nemocničnom informačnom systéme Univerzitnej nemocnice v Bratislave nie je možnosť jednoducho získať tabuľku obsahujúcu dáta jedného pacienta respektíve skupiny pacientov, tieto dát sa získavali ručne buď postupným kopírovaním jednotlivých dát nachádzajúcich sa v rôznych častiach informačného systému alebo ich hľadaním a následným prepisovaním z prepúšťacej správy.

Tento proces bolo potrebné opakovať pre každého jedného pacienta čo bolo časovo náročné a vyžadovalo si nezanedbateľné množstvo ľudskej práce.

Hlavným účelom tejto práce je vytvorenie softvéru ktorý pomôže výrazne zrýchliť a zjednodušiť získavanie týchto dát.

Tento softvér na svojom vstupe dostane prepúšťaciu správu pacienta a jeho krvné výsledky v podobe klasického textu a následne z týchto ne-štrukturalizovaných dát získava jednotlivé požadované dáta pričom v prípade, že nejakú informáciu nenájde alebo zistí, že zistená hodnota nie je v očakávaných limitoch alebo, že sa v správe nachádza viacero hodnôt pre jednu informáciu tak užívateľovi oznámi o akú informáciu ide a aký je s ňou problém. 

Práca je rozdelená do štyroch kapitol. Prvá kapitola sa zameriava na štruktúru prepúšťacej správy a dáta ktoré sa z nej snažíme získať.

Druhá kapitola obsahuje problémy ktoré sa objavili pre jednotlivé získavané informácie a aké spôsoby riešenia sme sa rozhodli použiť.

V tretej sú riešené časti kódu ktoré priamo nesúvisia so získavaním dát ako je zápis dát do výslednej tabuľky všetkých pacientov, zapisovanie problematických dát do logovacieho súboru alebo grafické prostredie pre jednoduchšiu prácu so softvérom.

Posledná kapitola je... nejaká asi napríklad iné možnosti prístupu k problému abo také niečo... asi nejaké analýzy chybovosti.

%\begin{quote}
%Úvod je prvou komplexnou informáciou o práci, jej cieli, obsahu a štruktúre. Úvod sa 
%vzťahuje na spracovanú tému konkrétne, obsahuje stručný a výstižný opis 
%problematiky, charakterizuje stav poznania alebo praxe v oblasti, ktorá je predmetom 
%školského diela a oboznamuje s významom, cieľmi a zámermi školského diela. Autor 
%v úvode zdôrazňuje, prečo je práca dôležitá a prečo sa rozhodol spracovať danú tému. 
%Úvod ako názov kapitoly sa nečísluje a jeho rozsah je spravidla 1 až 2 strany.
%\end{quote}


