% !TeX spellcheck = sk_SK-Slovak
\chapter*{Úvod} % chapter* je necislovana kapitola
\addcontentsline{toc}{chapter}{Úvod} % rucne pridanie do obsahu
\markboth{Úvod}{Úvod} % vyriesenie hlaviciek

V nemocničnom informačnom systéme Univerzitnej nemocnice v Bratislave nie je možnosť jednoducho získať tabuľku obsahujúcu dáta jedného pacienta respektíve skupiny pacientov, tieto dáta sa získavali ručne buď postupným kopírovaním jednotlivých dát nachádzajúcich sa v rôznych častiach informačného systému alebo ich hľadaním a následným kopírovaním z prepúšťacej správy.

Tento proces bolo potrebné opakovať pre každého jedného pacienta čo bolo časovo náročné a vyžadovalo si nezanedbateľné množstvo ľudskej práce.

Hlavným účelom tejto práce je vytvorenie softvéru, ktorý pomôže výrazne zrýchliť a zjednodušiť získavanie týchto dát. Tento softvér na svojom vstupe dostane prepúšťaciu správu pacienta a jeho krvné výsledky v podobe textu ktorý je čiastočne generovaný nemocničným informačným systémom ale z veľkej časti je písaný lekárom a následne z týchto čiastočne štrukturalizovaných dát získava jednotlivé požadované informácie. Počas ich získavania si zároveň zapamätáva informáciu o prípadných problémoch ktoré sa pri získavaní vyskytli ako napríklad viac rôznych hodnôt pre danú informáciu, protirečiace informácie alebo úplne chýbajúca informácia. Výsledkom je asociatívne pole v ktorom sú ako kľúče použité názvy hľadaných informácii a text obsahujúci informáciu o nenájdených a problémových údajoch. 

\begin{figure}
	%vlozenie samotneho obrazku vycentrovaneho a vhodnej velkosti
	%obrazok je v subore images/cervik.png
	\centerline{\includegraphics[width=0.8\textwidth]{images/system_jedna_sprava}}
	%popis obrazku
	\caption[Fungovanie systému pre jednu správu]{Fungovanie systému pre jednu správu}
	%id obrazku, pomocou ktoreho sa budeme na obrazok odvolavat
	\label{obr:systemJedna}
\end{figure}
\newpage
Okrem spracovania jedného pacienta je tento systém schopný spracovať aj väčšiu množinu pacientov pričom v takomto prípade už nie je výsledkom asociatívne pole ale je ním tabuľka v ktorej sú získavané údaje stĺpcami a jednotlivý pacienti riadkami. Zároveň je výstupom aj logovací súbor ktorý pre každého spracovaného pacienta obsahuje informácia o ktorého pacienta ide a pri ktorých informáciách nastal počas získavania problém a ktoré sa nájsť nepodarilo.

\begin{figure}
	%vlozenie samotneho obrazku vycentrovaneho a vhodnej velkosti
	%obrazok je v subore images/cervik.png
	\centerline{\includegraphics[width=0.8\textwidth]{images/system_viac_sprav}}
	%popis obrazku
	\caption[Fungovanie systému pre množinu správ]{Fungovanie systému pre množinu správ}
	%id obrazku, pomocou ktoreho sa budeme na obrazok odvolavat
	\label{obr:systemJedna}
\end{figure}

Systém je konkrétne určený na získavanie základných informácii o pacientovi a jeho hospitalizácii ako sú jeho osobné údaje, výška, váha, liečba, krvné výsledky a tak ďalej. Zároveň však získava aj dáta špecifické pre pacientov s ochorením COVID-19 ako sú oxygenoterapia alebo výsledky testov na protilátky proti vírusu SARS-CoV-2.

Z hľadiska obsahu samotnej bakalárskej práce je jej text rozdelený do piatich kapitol. Prvá obsahuje opis a porovnanie dvoch najbežnejších metód získavanie informácie z textu, príklady už existujúcich softvéroch využívajúcich tieto metódy na získavanie medicínskych dát a ich rozdiely oproti nášmu systému. Druhá kapitola sa zameriava na štruktúru vstupných dát, čiže prepúšťacej správy a krvných výsledkov, a na informácie ktoré tieto dáta obsahuje a ktoré z nich sa snažíme získať. Tretia kapitola obsahuje spôsoby získavania jednotlivých informácii, respektíve skupín informácii, problémy ktoré sa pri tomto získavaní objavili a aké spôsoby riešenia sme sa rozhodli použiť. V štvrtej kapitole sa snažíme určiť chybovosť nášho softvéru.  Posledná kapitola je o modifikáciách softvéru aby bol jednoduchšie použiteľný a bol modifikovateľný pre iné podobné použitia.

%\begin{quote}
%Úvod je prvou komplexnou informáciou o práci, jej cieli, obsahu a štruktúre. Úvod sa 
%vzťahuje na spracovanú tému konkrétne, obsahuje stručný a výstižný opis 
%problematiky, charakterizuje stav poznania alebo praxe v oblasti, ktorá je predmetom 
%školského diela a oboznamuje s významom, cieľmi a zámermi školského diela. Autor 
%v úvode zdôrazňuje, prečo je práca dôležitá a prečo sa rozhodol spracovať danú tému. 
%Úvod ako názov kapitoly sa nečísluje a jeho rozsah je spravidla 1 až 2 strany.
%\end{quote}


