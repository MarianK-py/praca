% !TeX spellcheck = sk_SK-Slovak
\chapter{Chybovosť systému}

Táto kapitola je zameraná na analýzu presnosti a chybovosti nášho systému.

\section{Validačné dáta}

Z dôvodu malého množstva predspracovaných dát a problémoch s nimi popísaných v časti \ref{predSprac} sme sa rozhodli, že na účely validácia výsledkov nášho systému ručne pracujeme dáta o 100 pacientoch ktorý neboli použitý pri vytváraní regulárnych výrazov.  

Konkrétny postup ktorý sme zvolili bol následovný:

\begin{enumerate}
	\item Príprava súboru obsahujúceho vstupné informácie o 100 pacientoch
	\item Ručné spracovanie pacientov
	\item Spracovanie pacientov využitím nášho systému
	\item Vytvorenie kópie systémom spracovaných pacientov
	\item Vylepšenie systémom spracovaných pacientov pomocou logovacieho súboru 
	\item Porovnanie ručne získaných dát s dátami získanými pomocou nášho systému pôvodnými aj vylepšenými
	\item Určenie chybovosti nášho systému   
\end{enumerate}

Keďže naše ''validačné'' dáta boli získavané ručne nemáme istotu, že sú bezchybné, preto sme sa pre účely určenia chybovosti systému rozhodli každý údaj ktorý je totožný s údajom získaným našim systémom považovať za správny a každý údaj ktorý sa líši skontrolujeme aby sme určili či je to chybou nášho systému alebo či ide o ľudskú chybu.

\section{Časová náročnosť}

Každý so spôsobov spracovania pacientov bol rôzne časovo náročný, v prípade ručného spracovania trvalo spracovanie jedného pacienta približne 5 minút, čiže spracovanie 100 pacientov trvalo okolo 500 minúť čiže viac ako 8 hodín. Náš systém dokázal spracovať 100 pacientov za menej ako minútu konkrétne priemerný čas ktorý na 100 pacientov potreboval bol 40 sekúnd, následné ručné vylepšenie pomocou logovacieho súboru trvalo pri 100 pacientoch približne 30 minút.

\section{Chybovosť jednotlivých skupín získavaných}

Z dôvodu, že náročnosť získavania nie je rovnaká pre všetky získavané dáta ani chybovosť nie je rovnaká preto si najskôr prejdeme chybovosť nášho systému pre jednotlivé skupiny získavaných dát (skupiny sú rovnaké ako v kapitole \ref{zisk}).

\subsection{Osobné údaje pacienta a obdobie hospitalizácie}

V prípade údajov ako je meno a priezvisko pacienta jeho rodné číslo a dátumy prijatia a prepustenia sa neobjavila žiadna chyba. V prípade dĺžky hospitalizácie sa objavili 2 nezrovnalosti avšak ukázalo sa, že išlo o chyby pri ručnom spracovaní a systém vypočítal správne hodnoty. Pri informácii o veku pacienta sa objavilo pomerne veľké množstvo nezrovnalosti konkrétne 22 pričom toto bolo spôsobený tým, že pri ručnom získavaní bola využitá hodnota veku ktorú lekár napísal do správy a v prípade nášho systém bol vek vypočítaný na základe rodného čísla a dátumu prepustenia. Presnejšie išlo o to, že lekár s najväčšou pravdepodobnosťou počítal vek iba na základe roku narodenia a z tohto dôvodu do správy napísal vyšší vek aj napriek tomu, že pacient ešte v danom roku nemal narodeniny.

Kontrola s logovacím súborom nepriniesla žiadne zlepšenie. Pri získavaní týchto informácii vykazuje náš systém nulovú chybovosť.

\subsection{JIS a smrť}

V prípade určenia či bol pacient hospitalizovaný sa jednotke intenzívnej starostlivosti a či počas hospitalizácie umrel sa nevyskytol žiaden prípad kde by sa systém nezhodol s ručne získanou informáciou. Takže v tomto prípade to podľa výsledkov vyzerá takisto na nulovú chybovosť, to zároveň znamená, že ani v tomto prípade úprava použitím logovacieho súboru nijak nezlepšila výsledok. 

\subsection{Výška a váha}

Pri zisťovaní výšky a váhy sa objavilo dohromady 5 chýbajúcich údajov pri troch pacientoch, trikrát išlo o váhu a dvakrát o výšku v jednom prípade išlo o problém keď systém pri váhe prekvapila informácia ''cca'' na ktorú nevedel reagovať, v ďalšom prípade nastal problém, keď bolo príliš veľa medzier medzi hodnotami výšky a váha, čo spôsobilo, že systém to nebol schopný nájsť keďže z bezpečnostných dôvodov (aby nepovažoval informáciu nachádzajúcu sa ďalej za hľadanú). Posledným problémom bola situácia keď hľadané informácie boli v špecifickom tvare konkrétne ''cm[hodnota]/[hodnota]kg'' čo náš systém nebol schopný nájsť.

Všetky tieto problémy boli zapísané v logovacom súbore vďaka čomu pri úpravy výsledkov s jeho použitím boli eliminované všetky chyby.

Takže systém spravil 5 chýb pri 200 hodnotách (2 údaje, 100 pacientov) z čoho vyplýva chybovosť približne 2,5\%, avšak pri použití logovacieho súboru je táto chybovosť nulová.     

\subsection{Saturácia krvi kyslíkom pri prijatí}

Pri hľadaní informácie o saturácie krvi kyslíkom pri prijatí náš systém spravil 3 chyby, v dvoch prípadoch za správnu informáciu považoval tú ktorá hovorili o hodnote saturácie po pripojení na oxygenoterapiu keďže bola v správe zapísaná prioritne a informácia o saturáciu bez prídavného kyslíka bola napísaná až neskôr v správe. Jedna chyba bola zapríčinené neočakávaným slovom ''RZP'' nachádzajúcim sa pred hodnotou saturácie.

Pri vylepšení pomocou logovacieho súboru sa podarilo odstrániť chybu zapríčinenú slovom ''RZP'', avšak hodnoty ovplyvnené oxygenoterapiou sa opraviť nepodarilo. Práve kvôli takýmto prípadom boli úvahy o zmene prístupu pri hľadaní tejto informácie a nevyužívať hodnotu ktorá bola zapísaná ako tá hlavná pri prijatí pacienta ale využiť najnižšiu nájdenú hodnoty, tento spôsob vychádzala z úvahy, že po prijatí by vďaka liečbe nemala saturácia klesnúť (oxygenoterapia by mala byť nastavovaná tak aby takáto situácia nenastala), avšak pri vytváraní systém sa táto úvaha ukázala ako nesprávna a často vykazovala nesprávne hodnoty preto nebola nakoniec využitá.

Výsledná chybovosť systému bola teda pri saturácii 3\% a pri použití logovacieho súboru klesla na 2\%.

\subsection{Protilátky proti vírusu SARS-CoV-2 pri prijatí}

Získavanie výsledkov testov na protilátky proti vírusu SARS-CoV-2 bolo nesprávne pri štyroch pacientoch vo všetkých štyroch prípadoch išlo o problém keď výsledky testov boli zapísané v časti ''Epikríza'' (\ref{blokE}) pričom v tejto časti správy lekári často nenapísali k testu názov vírusu (daný názov iba vyplýval z kontextu) čo spôsobilo, že systém to nedokázal nájsť. V troch prípadoch to bol jediný test v celej správe čo spôsobilo, že systém nenašiel žiadne výsledky vďaka čomu to ich systém označil za problém a pri oprave pomocou logovacieho súboru sa tieto chyby odstránili. V jednom prípade správa obsahovala niekoľko testov na protilátky ktoré boli pozitívne v oboch protilátkach avšak v epikríze bola informácie o teste s negatívnym výsledok pri protilátky typu IgG čo spôsobilo chybu ktorá nebola do logovacieho súboru zapísaná.

Takže chybovosť systému samotného bola 4\% a chybovosť po vylepšení logacím súborom klesla na 1\%.   

\subsection{Oxygenoterapia}

\subsection{Lieky}

Pri určovaní či pacient užíva vybrané lieky sa objavili 3 chyby. V jednom prípade lekár spravil chybu a napísal ''Dexamentazon'' namiesto ''Dexametazon'' čo bola chyba ktorú systém neočakával. Ďalšie dve chyby vznikli z dôvodu, že lekár do časti ''Anamnéza'' (\ref{blokB}) informácia ''Ivermektín neužíva'', takáto situácia sa počas vytvárania softvéru neobjavila a preto ju systém nebol schopný správne vyhodnotiť.

V tomto prípade logovací súbor nijak nepomohol s opravou chýb preto je chybovosť z pohľadu počtu pacientov s chybou 3\% (3 zo 100) avšak chybovosť z pohľadu počtu nesprávne určených hodnôt je iba 0.5\% (3 zo 600 (6 liekov, 100 pacientov)).

\subsection{Choroby pacienta}

\subsection{Krvné výsledky}