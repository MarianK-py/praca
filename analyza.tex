% !TeX spellcheck = sk_SK-Slovak
\chapter{Analýza výsledkov a chýb}

Táto kapitola je zameraná na analýzu presnosti a chybovosti nášho systému.

\section{Validačné dáta}

Z dôvodu malého množstva predspracovaných dát a problémoch s nimi popísaných v časti \ref{predSprac} sme sa rozhodli, že na účely validácia výsledkov nášho systému ručne pracujeme dáta o 100 pacientoch ktorý neboli použitý pri vytváraní regulárnych výrazov.  

Konkrétny postup ktorý sme zvolili bol následovný:

\begin{enumerate}
	\item Príprava súboru obsahujúceho vstupné informácie o 100 pacientoch
	\item Ručné spracovanie pacientov
	\item Spracovanie pacientov využitím nášho systému
	\item Vytvorenie kópie systémom spracovaných pacientov
	\item Vylepšenie systémom spracovaných pacientov pomocou logovacieho súboru 
	\item Porovnanie ručne získaných dát s dátami získanými pomocou nášho systému pôvodnými aj vylepšenými
	\item Určenie chybovosti nášho systému   
\end{enumerate}

Keďže naše ''validačné'' dáta boli získavané ručne nemáme istotu, že sú bezchybné, preto sme sa pre účely určenia chybovosti systému rozhodli každý údaj ktorý je totožný s údajom získaným našim systémom považovať za správny a každý údaj ktorý sa líši skontrolujeme aby sme určili či je to chybou nášho systému alebo či ide o ľudskú chybu.