% !TeX spellcheck = sk_SK-Slovak
\chapter{Podobné práce}

Aj napriek moderným nemocničným informačným
systémom je stále veľké množstvo nemocničných záznamov v
podobe čistého alebo čiastočne štrukturalizovaného 
textu z ktorého je ručné získavanie dát časovo náročné.
Preto sa na to využívajú automatizované systémy
ktoré zväčša fungujú na jednom z dvoch princípov 
respektíve na kombinácii oboch. Tými prístupmi
sú regulárne výrazy a metódy strojového učenia
určené na spracovanie prirodzeného jazyka.

Obe tieto prístupy majú svoje výhody a nevýhody \cite{nlpAndRegex}. 
Výhodou regulárnych výrazov je ich presnosť a 
transparentnosť čiže možnosť vidieť a upravovať
vnútorné fungovanie programu čiže jednotlivé výrazy
hľadajúce konkrétne informácie ale ich nevýhodou je,
že hľadané výrazy treba ručne vytvárať a vylepšovať
čo je často náročné, špecifické pre určitú oblasť
a v prípade komplikovaných výrazov je aj komplikovaná
údržba. Na druhej strane v prípade použitia niektorej
z metód strojového učenia môže často stačiť použiť už 
existujúcu metódu spracúvajúcu prirodzený jazyk mierne
ju modifikovať pre konkrétne použitie a natrénovať
model na predpripravených dátach avšak tieto modely
často nie sú až tak presné ako regulárne výrazy 
a zároveň v prípade nájdenia častej chyby alebo 
nutnosti modifikácie hľadaných dát (pridanie alebo 
odobranie získavanej informácie) je nutné model 
upraviť a celý nanovo pretrénovať a validovať aj už
skôr validované časti.

Medzi už existujúce systémy využívajúce regulárne
výrazy patrí napríklad systém HEDEA \cite{hedea} čiže 
Healthcare Data Extraction and Analysis ktorého
autormi sú Anshul Aggarwal, Sunita Garhwal a Ajay Kumar 
a bol vyvinutý na získavanie Indických medicínskych dát.
Systém využitím regulárnych výrazov nájde požadovanú informáciu 
v texte následne ju spracuje do podoby dvojica získavaný
údaj, výsledok (napríklad (diabetes, áno) alebo (výška, 170cm))
a tento údaj je následne zapísaný do databázy 
k konkrétnemu pacientovi na základe jeho identifikačného
čísla ktorý sa na každom spracovávanom texte nachádzať. 
Podobne ako v našom prípade je hlavnou úlohou tohto 
softvéru získavať medicínske dáta z čiastočne štrukturalizovaných
vstupných dát čiže lekárskych správ a výsledkov testov.
Hlavné rozdiely oproti nášmu systému sú, že systém HEDEA
sa snaží získavať základné dáta o pacientovi ako sú osobné
údaje, výška, váha, tlak, základné krvné výsledky a tak ďalej 
zatiaľ čo my získavame okrem týchto dát aj dáta 
špecifické pre pacientov s ochorením COVID-19 ako
napríklad typ oxygenoterapie alebo výsledky testov
na protilátky proti vírusu SARS-CoV-2, zároveň
je tento model určený na spracovávanie Indických dát
takže je vytvorený pre dáta písané v oficiálnom jazyku
Indie v tomto prípade angličtinu zatiaľ čo náš model
je vytvorený pre dáta v slovenčine.

Systémom využívajúcim metódu strojového učenia na
spracovávanie prirodzeného jazyka je napríklad 
systém ktorý vytvorili Fette a spol. na Univerzite
vo Würzburgu \cite{infExtGer} využívajúci metódu učenia s učiteľom 
s názvom Conditional random field \cite{CRF} ktorej úlohou
je označiť jednotlivé slová respektíve viacslovné
pomenovania a následne pomocou metódy 
Keyword Matching with Terminology based disambiguation
prepojiť nájdené slová a viacslovné pomenovania s databázou
odborných pojmov tak, že ak je prepojenie jednoznačné
použije ho a ak je nejednoznačné hľadá v okolí slova
slovo s jednoznačným prepojením ktoré bližšie určí
prepojenie nejednoznačného slova. Okrem samotného spôsobu
získavanie dát je v porovnaní s našim systémom rozdiel 
aj v prioritách pri získavaní dát keďže náš systém 
je vytvorený na čo najväčšiu presnosť pri získavaní
dát špecificky z prepúšťacích správ zatiaľ čo ich 
systém je vytvorený tak aby mohol byť natrénovaný na získavanie
dát z rôznych typov medicínskych dokumentov či už
lekárskych správ, výsledkov testov alebo klinických 
štúdii a takisto platí, že celý systém je vytvorený 
pre iný jazyk ako náš systém v tomto prípade ide o 
nemčinu. 

Prístup ktorý kombinuje metódu strojového učenia 
s regulárnymi výrazmi využili v svojom systéme
Cui a kol. \cite{CHA} ktorý využili metódu s názvom 
Constructive heuristic ktorej úlohou nebolo
priamo hľadanie získavaných informácii v texte ale
generovanie čo najlepších regulárnych výrazov
na to určených. Tento algoritmus začína s
prazdnou množinou regulárnych výrazov a 
následne iteratívne túto množinu rozširuje
a upravuje kým nie je splnená ukončovacia
podmienka. Výhodou tohto prístupu oproti
bežným metódam strojového učenia je, 
že na konci trénovania má uživateľ množinu
regulárnych výrazov kroré môže ďalej upravovať
a nie "čiernu skrinku" ktorej vnútornému fungovaniu
nerozumie, oproti len použitiu regulárnych výrazov
má výhodu, že nie je nutné ich vymýšľať
od začiatku ale stačí iba výstup mierne upraviť.
Hlavným rozdielom oproti nášmu systému je to, 
že hlavnou ulohou ich systém nie je priame získavanie
dát z medicínskej dokumentácie ale generovanie 
regulárnych výrazov ktoré je po na takýto problém
možné použiť.

