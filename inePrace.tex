% !TeX spellcheck = sk_SK-Slovak
\chapter{Podobné práce}

Aj napriek moderným nemocničným informačným
systémom je stále veľké množstvo nemocničných záznamov v
podobe čistého alebo čiastočne štrukturalizovaného 
textu z ktorého je ručné získavanie dát časovo náročné.
Preto sa na to využívajú automatizované systémy
ktoré zväčša fungujú na jednom z dvoch princípov, 
respektíve na kombinácii oboch. Tými prístupmi
sú regulárne výrazy a metódy strojového učenia
určené na spracovanie prirodzeného jazyka.

Obe tieto prístupy majú svoje výhody a nevýhody \cite{nlpAndRegex}. 
Výhodou regulárnych výrazov je ich presnosť a 
transparentnosť čiže možnosť vidieť a upravovať
vnútorné fungovanie programu, presnejšie možnosť 
upravovať jednotlivé regulárne výrazy
hľadajúce konkrétne informácie, ale ich nevýhodami sú,
že všetky regulárne výrazy ktoré softvér využíva 
treba ručne vytvárať a vylepšovať
čo je často náročné, väčšinou sú špecifické pre určitú oblasť
pre ktorú je softvér vytváraný čo komplikuje využívanie
regulárnych výrazov ktorých správne fungovanie bolo už 
otestované v iných softvéroch 
a v prípade komplikovaných výrazov je aj ich 
upravovanie a vylepšovanie komplikované. 
Na druhej strane v prípade použitia niektorej
z metód strojového učenia je často možné využiť už 
existujúcu metódu spracúvajúcu prirodzený jazyk a
modifikovať ju pre konkrétne použitie a natrénovať
model na predpripravených dátach, avšak aj tento prístup
má nevýhody ako napríklad, že natrénovaný model
často nie sú až tak presný ako dobre nastavené
regulárne výrazy, pre zvýšenie presnosti je často nutné
zväčšiť množstvo ručne spracovať dát určených na trénovanie 
modelu a zároveň v prípade nájdenia častej chyby alebo 
nutnosti modifikácie hľadaných dát (pridanie alebo 
odobranie získavanej informácie) je nutné model 
upraviť a celý nanovo pretrénovať a validovať aj už
skôr validované časti.

Medzi už existujúce systémy využívajúce regulárne
výrazy patrí napríklad systém HEDEA \cite{hedea} čiže 
Healthcare Data Extraction and Analysis ktorého
autormi sú Anshul Aggarwal, Sunita Garhwal a Ajay Kumar 
a bol vyvinutý na získavanie Indických medicínskych dát.
Systém využitím regulárnych výrazov hľadá každú jednu
získavanú informáciu tak, že hľadá v texte kľúčové slovo
označujúce požadovanú informáciu, následne ak by mala 
k danej informácií existovať aj konkrétna hodnota
(či už číselná alebo slovná) tak hľadá túto hodnotu v okolí
kľúčového slova a následne túto informáciu aj 
s jej hodnotou zapíše do databázy k konkrétnemu 
pacientovi na základe jeho identifikačného
čísla ktoré sa na každom spracovávanom texte musí nachádzať. 
Podobne ako v našom prípade je hlavnou úlohou tohto 
softvéru získavať medicínske dáta z čiastočne štrukturalizovaných
vstupných dát čiže lekárskych správ a výsledkov testov.
Hlavné rozdiely oproti nášmu systému sú, že systém HEDEA
sa snaží získavať iba základné dáta o pacientovi ako sú osobné
údaje, výška, váha, tlak, základné krvné výsledky a 
prekonané ochorenia inými slovami jeho účelom je vytvoriť
databázu obsahujúcu anamnézy jednotlivých pacientov
ktorú môže využiť lekár pri diagnostike daného pacienta
zatiaľ čo my získavame okrem týchto dát aj dáta 
špecifické pre pacientov s ochorením COVID-19 ako
napríklad typ oxygenoterapie alebo výsledky testov
na protilátky proti vírusu SARS-CoV-2 a našou snahou je 
vytvoriť tabuľku s dátami o ochorení COVID-19 ktorá
sa dá využiť na analýzu rizikových faktorov a liečby
tohto ochorenia, zároveň systém HEDEA určený na 
spracovávanie Indických dát
takže je vytvorený pre dáta písané v oficiálnom jazyku
Indie v tomto prípade angličtinu zatiaľ čo náš model
je vytvorený pre dáta v slovenčine.

Systémom využívajúcim metódu strojového učenia na
spracovávanie prirodzeného jazyka je napríklad 
systém ktorý vytvorili Fette a kol. na Univerzite
vo Würzburgu \cite{infExtGer} ktorý využívajúci metódu učenia s 
učiteľom s názvom Conditional random field ktorej úlohou
je označiť jednotlivé slová respektíve viacslovné
pomenovania \cite{CRF} a následne pomocou metódy 
Keyword Matching with Terminology based disambiguation
prepojiť nájdené slová a viacslovné pomenovania s databázou
odborných pojmov tak, že v prípade jednoznačného prepojenie 
považuje danú informáciu za klasifikovanú a v
prípade nejednoznačného prepojenia (dané slovo môže byť časťou
rôznych informácii napríklad v prípade číselnej hodnoty nevieme
bez ďalšej informácie určiť k čomu patrí) hľadá v okolí označeného 
slova iné označené slovo s jednoznačným prepojením ktoré bližšie 
určí prepojenie nejednoznačného slova. Okrem samotného spôsobu
získavanie dát je v porovnaní s našim systémom rozdiel 
aj v prioritách pri získavaní dát keďže náš systém 
je vytvorený na čo najväčšiu presnosť pri získavaní
dát špecificky z prepúšťacích správ zatiaľ čo ich 
systém je vytvorený tak aby ho bolo možné byť natrénovaný na 
získavanie dát z rôznych typov medicínskych dokumentov či už
lekárskych správ, výsledkov testov alebo klinických 
štúdii pričom jedinou podmienkou je aby sa všetky získavané údaje 
nachádzali v databáze odborných pojmov a takisto platí, 
že celý systém je vytvorený 
pre iný jazyk ako náš systém v tomto prípade ide o 
nemčinu. 

Prístup ktorý kombinuje metódu strojového učenia 
s regulárnymi výrazmi využili v svojom systéme
Cui a kol. \cite{CHA} ktorý využili metódu s názvom 
Constructive heuristic ktorej úlohou nebolo
priamo hľadanie získavaných informácii v texte ale
generovanie čo najlepších regulárnych výrazov
na to určených. Tento algoritmus začína s
prázdnou množinou regulárnych výrazov a 
následne iteratívne túto množinu rozširuje
a upravuje kým nie je splnená ukončovacia
podmienka \cite{conHeu}. Výhodou tohto prístupu oproti
bežným metódam strojového učenia je, 
že na konci trénovania má užívateľ množinu
regulárnych výrazov ktoré môže ďalej upravovať
a nie "čiernu skrinku" ako v prípade bežnej metódy strojového
učenia, ktorej vnútornému fungovaniu je pre človeka
nepochopiteľné. Oproti len použitiu regulárnych výrazov
má výhodu, že nie je nutné regulárne výrazy vymýšľať
od začiatku ale stačí iba výstup mierne upraviť.
Nevýhodou je, že pochopenie a upravenie regulárnych
výrazov je síce možné ale môže to byť pomerne náročné
keďže ide o počítačom generované regulárne výrazy
ktoré aj napriek tomu, že fungujú rovnako dobre môžu
sa výrazne líšiť od toho čo by napísal človek.  
Hlavným rozdielom oproti nášmu systému je to, 
že hlavnou úlohou ich systém nie je priame získavanie
dát z medicínskej dokumentácie ale generovanie 
regulárnych výrazov ktoré je po na takýto problém
možné použiť.

