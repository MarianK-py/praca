% !TeX spellcheck = sk_SK-Slovak
\chapter{Metódy získavania dát a podobné systémy}

Aj napriek moderným nemocničným informačným
systémom je stále veľké množstvo nemocničných záznamov v
podobe čistého alebo čiastočne štrukturovaného 
textu z ktorého je ručné získavanie dát časovo náročné.
Preto sa na to využívajú automatizované systémy ktoré z týchto textov dokážu získať potrebnú informáciu. 

Táto kapitola sa zameriava na zoznámenie sa s dvomi najčastejšími prístupmi na takéto získavanie dát. Zároveň táto kapitola obsahuje tri príklady systémov ktoré tieto prístupy využívajú na získavanie medicínskych dát a ich rozdiely oproti nášmu systému.

\section{Metódy získavania dát z textu}


Väčšina systémov určených na získavanie dát z textu funguje na jednom z dvoch princípov, 
respektíve na kombinácii oboch. Týmito prístupmi
sú regulárne výrazy a metódy strojového učenia
určené na spracovanie prirodzeného jazyka.

% pridat kratke vysvetlenie fungovania danych metod
Metóda regulárnych výrazov využíva na hľadanie informácii v texte
špeciálne kódované reťazce znakov, takzvané regulárne výrazy, ktoré slúžia ako vzor. 
Systém hľadá v texte časti ktoré sa zhodujú so vzorom a z 
nájdených častí sa následne získava konkrétna hľadaná informácia.

Metóda strojového učenia využíva algoritmy učenia umelej inteligencie s učiteľom. Systém dostane trénovacie dáta, čiže množinu textov z ktorých má získavať informácie a množiny správne získaných informácii, a pomocou týchto dát sa učí hľadať požadované informácie v textoch. 

Oba tieto prístupy majú svoje výhody a nevýhody \cite{nlpAndRegex}. 
Výhodou regulárnych výrazov je ich presnosť a 
transparentnosť čiže možnosť vidieť a upravovať
vnútorné fungovanie programu, presnejšie možnosť 
upravovať jednotlivé regulárne výrazy
hľadajúce konkrétne informácie. Medzi nevýhody tohto prístupu patrí
napríklad fakt, že všetky regulárne výrazy ktoré softvér využíva 
treba ručne vytvárať a vylepšovať
čo je často komplikované. Zároveň sú tieto regulárne výrazy väčšinou špecifické pre určitú doménu
pre ktorú je softvér vytváraný, čo často znemožňuje využívanie
regulárnych výrazov, ktorých správne fungovanie bolo už 
otestované v softvéroch z inej domény.  
Na druhej strane v prípade použitia niektorej
z metód strojového učenia je výhodou to, že je často možné využiť už 
existujúcu metódu na spracovanie prirodzeného jazyka a iba ju
modifikovať pre konkrétne použitie a natrénovať
model na predpripravených dátach. Avšak aj tento prístup
má nevýhody. Natrénovaný model často nie je až tak presný 
ako dobre nastavené regulárne výrazy. Pre zvýšenie presnosti je často nutné zväčšiť množstvo ručne spracovať dát určených na trénovanie modelu čo môže byť časovo náročné. Zároveň v prípade nájdenia častej chyby alebo nutnosti modifikácie hľadaných dát (pridanie alebo odobranie získavanej informácie) je nutné model upraviť a celý nanovo pretrénovať a validovať aj už skôr validované časti.

Pre náš systém sme si vybrali metódu regulárnych výrazov. Hlavným dôvodom tohto výberu bolo malé množstvo dát použiteľných na trénovanie modelu strojového učenia. 

\section{Podobné systémy}

Teraz sa pozrieme na 3 príklady systémov určených na získavanie medicínskych dát z textu. Pričom každý z týchto systémov funguje na inom princípe.

Príkladom systému využívajúceho regulárne výrazy je Healthcare Data Extraction and Analysis (HEDEA) \cite{hedea}, ktorého
autormi sú Anshul Aggarwal, Sunita Garhwal a Ajay Kumar, 
a bol vyvinutý na získavanie Indických medicínskych dát.
Systém využitím regulárnych výrazov hľadá každú jednu
získavanú informáciu tak, že hľadá v texte kľúčové slovo
označujúce požadovanú informáciu, následne ak by mala 
k danej informácií existovať aj konkrétna hodnota
(či už číselná alebo slovná) tak hľadá túto hodnotu v okolí
kľúčového slova a nakoniec túto informáciu aj 
s jej hodnotou zapíše do databázy ku konkrétnemu 
pacientovi na základe jeho identifikačného
čísla ktoré sa na každom spracovávanom texte musí nachádzať. 
Podobne ako v našom prípade je hlavnou úlohou tohto 
softvéru získavať medicínske dáta z čiastočne štrukturalizovaných
vstupných dát čiže lekárskych správ a výsledkov testov.
Hlavné rozdiely oproti nášmu systému sú, že systém HEDEA
sa snaží získavať iba základné dáta o pacientovi ako sú osobné
údaje, výška, váha, krvný tlak, základné krvné výsledky a 
prekonané ochorenia. Inými slovami jeho účelom je vytvoriť
databázu obsahujúcu anamnézy jednotlivých pacientov
ktorú môže využiť lekár pri diagnostike,
zatiaľ čo my získavame okrem týchto dát aj dáta 
špecifické pre pacientov s ochorením COVID-19 ako
napríklad typ oxygenoterapie alebo výsledky testov
na protilátky proti vírusu SARS-CoV-2 a našou snahou je 
vytvoriť tabuľku s dátami o pacientoch ktorý trpeli ochorením COVID-19 ktorá
sa dá využiť na analýzu rizikových faktorov a účinnosti liečby
tohto ochorenia, zároveň systém HEDEA určený na 
spracovávanie Indických dát
takže je vytvorený pre dáta písané v oficiálnom jazyku
Indie v tomto prípade angličtinu zatiaľ čo náš model
je vytvorený pre dáta v slovenčine.

Systémom využívajúcim metódu strojového učenia na
spracovávanie prirodzeného jazyka je napríklad 
systém ktorý vytvorili Fette a kol. na Univerzite
vo Würzburgu \cite{infExtGer} ktorý využívajúci metódu učenia s 
učiteľom s názvom Conditional random field ktorej úlohou
je označiť jednotlivé slová, respektíve viacslovné
pomenovania \cite{CRF} pričom tieto slová a viacslovné pomenovania sú následne pomocou metódy 
Keyword Matching with Terminology based disambiguation
hľadané v databáze
odborných pojmov tak, že v prípade jednoznačnej zhody 
považuje systém danú informáciu za klasifikovanú a v
prípade nejednoznačného prepojenia (dané slovo môže byť časťou
rôznych informácií napríklad v prípade číselnej hodnoty nevieme
bez ďalšej informácie určiť k čomu patrí) hľadá v okolí označeného 
slova iné označené slovo s jednoznačnou zhodou ktoré bližšie 
určí význam nejednoznačného slova. Okrem samotného spôsobu
získavanie dát je v porovnaní s našim systémom rozdiel 
aj v prioritách pri získavaní dát keďže náš systém 
je vytvorený na čo najväčšiu presnosť pri získavaní
dát špecificky z prepúšťacích správ zatiaľ čo tento 
systém je vytvorený tak aby ho bolo možné byť natrénovaný na 
získavanie dát z rôznych typov medicínskych dokumentov či už
lekárskych správ, výsledkov testov alebo klinických 
štúdii pričom jedinou podmienkou je aby sa všetky získavané údaje 
nachádzali v databáze odborných pojmov. Ďalším rozdielom je, 
že celý systém je vytvorený 
pre iný jazyk ako náš systém v tomto prípade ide o 
nemčinu. 

Prístup ktorý kombinuje metódu strojového učenia 
s regulárnymi výrazmi využili v svojom systéme
Cui a kol. \cite{CHA} ktorý využili metódu s názvom 
Constructive heuristic ktorej úlohou nie je
priame hľadanie získavaných informácii v texte ale
generovanie čo najlepších regulárnych výrazov
na to určených. Tento algoritmus začína s
prázdnou množinou regulárnych výrazov a 
následne iteratívne túto množinu rozširuje
a upravuje kým nie je splnená ukončovacia
podmienka \cite{conHeu}. Výhodou tohto prístupu oproti
bežným metódam strojového učenia je, 
že na konci trénovania má užívateľ množinu
regulárnych výrazov ktoré môže ďalej upravovať
a nie ''čiernu skrinku'' ako v prípade bežnej metódy strojového
učenia, ktorej vnútornému fungovaniu je pre človeka
nepochopiteľné. Oproti použitiu len regulárnych výrazov
má výhodu, že nie je nutné regulárne výrazy vymýšľať
od začiatku ale stačí iba výstup mierne upraviť.
Nevýhodou je, že pochopenie a upravenie regulárnych
výrazov je síce možné ale môže to byť pomerne náročné
keďže ide o počítačom generované regulárne výrazy
ktoré aj napriek tomu, že fungujú rovnako dobre môžu
sa výrazne líšiť od toho čo by napísal človek.  
Hlavným rozdielom oproti nášmu systému je to, 
že hlavnou úlohou ich systém nie je priame získavanie
dát z medicínskej dokumentácie ale generovanie 
regulárnych výrazov ktoré je po na takýto problém
možné použiť.

