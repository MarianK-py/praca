% !TeX spellcheck = sk_SK-Slovak
\chapter{Podobné práce}

Aj napriek moderným nemocničným informacným
systémom je stále veľké množstvo nemocničných záznamov v
podobe čistého alebo čiastične štrukturalizovaného 
textu z ktorého je ručné získavanie dát časovo náročné.
Preto sa na to využívajú automatizované systémy
ktoré zväčša fungujú na jednom z dvoch princípov 
respektíve na kombinácii oboch. Tými prístupmi
sú regulárne výrazy a metódy strojového učenia
určené na spracovanie prirodzeného jazyka.

Obe tieto prístupy majú svoje výhody a nevýhody [0]. 
Výhodou regulárnych výrazov je ich presnosť a 
transparentnosť čiže možnosť vidieť a upravovať
vnútorné fungovanie programu ale ich nevýhodov je,
že hľadané výrazy treba ručne vytvárať a vylepšovať
čo je často náročné, špecifické pre určitú oblasť
a v prípade komplikovaných výrazov je aj komplikovaná
údržba. Na druhej strane 

Získavanie dát z textu pomocou regulárnych výrazov 
sa v medicínskych aplikáciach často využíva keďže,
narozdiel on iných spôsobou získavania dát z textu
akými sú metódy strojového učenia na spracovanie
prirodzeného jazyka sú často dosť presné a zároveň
ich fungovanie je transparentné takže v prípade 
problému so získavanými dátami alebo v prípade 
potreby pridania alebo odobrania získavanej informácie
je úprava jednoduchá a spočíva zväčša iba v modifikácii,
pridaní alebo odstránení regulárneho výrazu. 

Medzi už existujúce systémy využívajúce regulárne
výrazy patrí napriklad systém HEDEA [1] čiže 
Healthcare Data Extraction and Analysis ktorého
autormi sú Anshul Aggarwal, Sunita Garhwal a Ajay Kumar 
a bol vyvinutý na získavanie Indických medicínskych dát.
Podobne ako v našom prípade je hlavnou úlohou tohto 
softvéru získavať medicínske dáta z čiatočne štrukturalizovaných
vstupných dát čiže lekárskych správ a výsledkou testou.
Hlavné rozdiely oproti nášmu systému sú, že systém HEDEA
sa snaží získavať základné dáta o pacientovi ako sú osobné
údaje, výška, váha, tlak, základné krvné výsledky a tak ďalej 
zatiaľ čo my získavame okrem týchto dát aj dáta 
špecifické pre pacientov s ochorením COVID-19 ako
napríklad typ oxygenoterapie alebo výsledky testov
na pritolátky proti vírusu SARS-CoV-2, zároveň
je tento model určený na spracovávanie Indických dát
takže je vytvorený pre dáta písané v oficialnom jazyku
Indie v tomto prípade algličtinu zatiaľ čo náš model
je vytvorený pre dáta v slovenčine.


[0] Cascaded Data Mining Methods for Text Understanding, with medical case study
[1] HEDEA: A Python Tool for Extracting and Analysing Semi-structured Information from Medical Records